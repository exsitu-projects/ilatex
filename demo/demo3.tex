\documentclass[11pt]{article}

\usepackage[T1]{fontenc}
\usepackage[utf8]{inputenc}
\usepackage[english]{babel}
\usepackage{xspace}
\usepackage{parskip}
\usepackage{amsmath}
\usepackage{amsfonts}
\usepackage{amssymb}
\usepackage{enumitem}
\usepackage[dvipsnames]{xcolor}
\usepackage{graphicx}
\usepackage{subcaption}
\usepackage{booktabs}
\usepackage{float}
\usepackage{array}
\usepackage[
    paper = a4paper,
    margin = 2.5cm,
    top = 2cm,
    bottom = 2cm
]{geometry}
\usepackage[colorlinks = true]{hyperref}
\usepackage{ilatex}


%%%%%%%%%%%%%%%%%%%%%%%%%%%%%%%%%%%%%%%%%%%%%%%%%%%%%

\newcommand{\iLaTeX}{i\LaTeX{}\xspace}

\title{A test file for \iLaTeX}
\author{Camille Gobert (\url{camille.gobert@ens.fr})}
\date{\today}

\linespread{1.05}


%%%%%%%%%%%%%%%%%%%%%%%%%%%%%%%%%%%%%%%%%%%%%%%%%%%%%

% You can view the result on Overleaf:
% https://www.overleaf.com/read/cvhwtykfnzpn

\begin{document}
    % Title
    \maketitle
    
    % Body
    Lorem ipsum dolor sit amet.
    Text \emph{accentu\'e} should be supported.
    
    \textbf{This is bold text.}
    \textit{This is italic text.}
    
    This is a \href{https://www.lri.fr/}{hyperlink} to LRI's website.
    
    % Example of some maths
    \paragraph{}
    Inline maths such as $x^2$ are supported, as well as 
    $$e^{i\pi} + 1 = 0$$
    
    % Example of a table
    \begin{table}[h]
        \centering
        \ilatex{\begin{tabular}{rl}
            \toprule
            \textbf{Symbol} & \textbf{Value} \\
            \midrule
            $x$ & 1 \\
            $y$ & 2 \\
            $z$ & 3 \\
            \bottomrule
        \end{tabular}}
        \caption{Random table.}
        \label{tab:random}
    \end{table}

    % Example of a figure
    \begin{figure}[h]
        \centering
        \ilatex{\includegraphics[width=224px, height=296px, trim=29px 163px 126px 73px, clip]{img/rabbit.jpg}}
        \caption{Picture of a rabbit.}
        \label{fig:rabbit}
    \end{figure}

    \newpage
    This text should appear on the second page.

    % A larger picture on the second page
    \begin{figure}
        \centering
        \begin{subfigure}[b]{.5\textwidth}
            \centering
            \ilatex{\includegraphics[width=90bp]{img/cat.jpg}}
            \caption{Picture of a cat.}
        \end{subfigure}%
        \begin{subfigure}[b]{.5\textwidth}
            \centering
            \ilatex{\includegraphics[width=4cm]{img/cat-drawing.png}}
            \caption{Drawing of a cat.}
        \end{subfigure}
        \caption{Two kinds of images in a single picture.}
        \label{fig:cats}
    \end{figure}

    \begin{figure}
        \centering
        \begin{subfigure}[b]{.5\textwidth}
            \centering
            \ilatex{\includegraphics[width=6cm]{img/cat.jpg}}
            \caption{Normal picture.}
        \end{subfigure}%
        \begin{subfigure}[b]{.5\textwidth}
            \centering
            \ilatex{\includegraphics[width=142px, height=142px, trim=64px 103px 88px 49px, clip]{img/cat.jpg}}
            \caption{Zoomed \& trimmed picture.}
        \end{subfigure}
        \caption{Some more cats.}
        \label{fig:more-cats}
    \end{figure}

    % A more complex table example taken from
    % https://www.overleaf.com/learn/latex/tables
    % \begin{table}[h]
    %     \centering
    %     \ilatex{\begin{tabular}{|p{3cm}||p{3cm}|p{3cm}|p{3cm}|}
    %         \hline
    %         \multicolumn{4}{|c|}{Country List} \\
    %         \hline
    %         Country Name or Area Name & ISO ALPHA 2 Code & ISO ALPHA 3 Code & ISO numeric Code \\
    %         \hline
    %         Afghanistan & AF & AFG & 004 \\
    %         Aland Islands & AX & ALA & 248 \\
    %         Albania & AL & ALB & 008 \\
    %         Algeria & DZ & DZA & 012 \\
    %         American Samoa & AS & ASM & 016 \\
    %         Andorra & AD & AND & 020 \\
    %         Angola & AO & AGO & 024 \\
    %         \hline
    %         \end{tabular}}
    %     \caption{Another meaningless table.}
    %     \label{tab:meaningless}
    % \end{table}
\end{document}
