\section{Interviews d'utilisateurs de \LaTeX{}}
\label{sec:interviews}

Afin de mieux comprendre de quelles façons les RII pourraient améliorer l'édition de langages de description de document, nous avons interviewé 11 utilisateurs de \LaTeX{} et réalisé une analyse thématique des difficultés que ceux-ci rencontrent.
Dans cette section, nous présentons la méthodologie utilisée et les résultats obtenus.
À notre connaissance, aucune étude similaire n'a déjà été publiée.


\subsection{Méthodologie}

\subsubsection{Participants}
Nous avons interviewé 11 participants (5~femmes et 6~hommes, âgés de~21 à~40 ans).
La plupart sont des étudiants de Master (8/11) ou des personnes issues du milieu universitaire, spécialisés dans des domaines très divers.
Les participants ont été recrutés via un mailing interne au laboratoire et un message sur un groupe Facebook d'une grande école.
Ils n'ont reçu aucune compensation pour leur participation.
Chacun d'entre eux avait utilisé \LaTeX{} au cours des semaines ou des mois précédant leur interview.
Les participants ont l'habitude d'utiliser Overleaf (5/11) ou Texmaker (5/11) pour rédiger des documents \LaTeX{}, mais plusieurs d'entre eux ont également utilisé ou essayé au moins un éditeur alternatif parmi TeXstudio, TeXworks, Kyle et Gedit --- sur Windows, MacOS ou Linux.
En outre, la majorité d'entre eux utilise également des logiciels de traitement de texte.
Leur expertise avec \LaTeX{} est très variable\footnote{Les niveaux d'expertise sont définis de la façon suivante : les \emph{débutants} ne se sentent pas à l'aise avec \LaTeX{} et ont besoin d'aide pour utiliser la plupart des fonctionnalités ; les utilisateurs \emph{intermédiaires} se sentent à l'aise pour écrire des documents simples avec quelques structures et mettre le texte en forme ; et les utilisateurs \emph{avancés} se sentent à l'aise pour inclure du contenu varié et créer leurs propres commandes. Chaque participant a été invité à auto-évaluer son niveau d'expertise, que nous avons ensuite adapté en fonction des fonctionnalités avec lesquelles il se sentait à l'aise ou non.}, mais la plupart des participants (8/11) sont des utilisateurs intermédiaires.
Des informations supplémentaires sur chaque participant peuvent être trouvés dans la {Table~\ref{tab:participants}}.

\begin{table*}[h]
    \centering
    \makebox[\textwidth]{\begin{itabular}{lllll}
        \toprule
        Participant & Profession & Domaine & Expertise & Principal document présenté \\
        \midrule
        P1 & Étudiante en Master & Biologie & Intermédiaire & Rapport de stage \\
        P2 & Doctorante & Visualisation de données & Intermédiaire & Thèse de doctorat \\
        P3 & Étudiant en Master & Écologie & Intermédiaire & Devoir à rendre \\
        P4 & Étudiante en Master & Géologie & Intermédiaire & Projet de hackathon \\
        P5 & Enseignant-chercheur & Informatique & Avancé & Article de mathématiques \\
        P6 & Étudiant en Master & Systèmes complexes & Intermédiaire & Sujet de thèse \\
        P7 & Étudiante en Master & Géochimie & Intermédiaire & Rapport de stage \\
        P8 & Étudiante en Master & Archéologie & Débutant & Rapport de stage \\
        P9 & Étudiant en Master & Informatique & Intermédiaire & Article d'informatique \\
        P10 & Professeur de lycée & Philosophie & Intermédiaire & Liste d'examens de philo. \\
        P11 & Étudiant en Master & Informatique & Avancé & Notes de cours de math. \\
        \bottomrule
    \end{itabular}}
    \caption{Détails sur les participants interviewés.}
    \label{tab:participants}
\end{table*}


\subsubsection{Procédure}
Toutes les interviews étaient semi-structurées, menées en français, et d'une durée moyenne d'une heure environ (entre 29 et 99 minutes).
La durée des interviews et le choix des questions ont été ajustés à l'aide de trois interviews pilotes de collègues doctorants et post-doctorants.

Nous avons commencé chaque interview en demandant à chaque participant de nous montrer le dernier document \LaTeX{} sur lequel il avait travaillé via un partage d'écran (comprenant à la fois le code \LaTeX{} et le PDF généré).
Nous leur avons tout d'abord posé plusieurs questions relatives au document présenté (\eg{} type de document, éditeur utilisé, etc).
Nous leur avons ensuite demandé de nous décrire les différents problèmes auxquels ils avaient fait face (qu'ils les aient résolus ou non) en se référant le plus possible au code du document (dans le double but de les aider à s'en souvenir et de rendre leurs explications plus claires pour nous).
Les interviews étant semi-structurées, nous avons utilisé une liste de questions prédéfinies pour guider les participants, mais ceux-ci étaient invités à parler librement de chaque problème rencontré.
Les participants étaient également invités à nous parler de problèmes rencontrés sur d'autres documents si cela leur semblait pertinent.
Nous avons conclu chaque interview avec quelques questions plus générales sur l'expérience de chaque participant avec \LaTeX{}.

\subsubsection{Données récoltées}
Toutes les interviews ont été enregistrées (audio et écran partagé) et retranscrites manuellement, incluant de courtes descriptions du contenu affiché à l'écran ou des interactions des participants lorsque cela était pertinent.
Nous avons également pris des notes manuscrites synthétisant les problèmes principaux et les solutions utilisées.

\subsubsection{Analyse des données}
\label{sssec:collecte-analyse-donnees}
Nous avons réalisé une analyse thématique~\cite{braun2019reflecting} à partir des données récoltées.
Le premier auteur a tout d'abord généré plus de 650 codes à partir des huit premières transcriptions, qu'il a ensuite classés dans près de 80 sous-thèmes.
Nous nous sommes alors servis de ces sous-thèmes afin de générer trois jeux de thèmes :
un premier jeu regroupant les sous-thèmes similaires (\ie{} des \emph{domain summaries}, dans le vocabulaire de Braun et Clarke) ;
un second jeu regroupant les problèmes similaires (indépendamment du contexte dans lequel ils sont apparus) ;
et un troisième jeu issu d'un mélange des deux premiers.

Ce dernier jeu a été élaboré à l'aide de discussions avec plusieurs chercheurs de l'équipe dans le but d'identifier les groupes de sous-thèmes les plus pertinents dans les deux premiers thèmes.
Nous avons également réécouté les trois interviews non-transcrites afin de nous assurer de leur inscription dans les cinq thèmes finaux, et nous en avons extrait plusieurs citations afin de renforcer et de contraster ce qui avait déjà été mis en évidence par les huit premières interviews.
Nous présentons le contenu des cinq thèmes finaux ci-dessous.




\subsection{Résultats}

\subsubsection{T1 --- Le code source doit être accessible et modifiable}

\paragraph{Les interfaces WYSIWYG sont limitées}
En dépit des difficultés que cela induit, la plupart des participants semblent à l'aise avec le paradigme consistant à décrire leur document dans un langage particulier ; et peu ont évoqué le désir d'un éditeur WYSIWYG pour \LaTeX{}.
P2 craint par exemple que déplacer des morceaux de texte dans Microsoft Word \citeparticipant{[fasse] du code dégueu\textasteriskcentered\textasteriskcentered\textasteriskcentered\textasteriskcentered\textasteriskcentered{} derrière}, tandis que P4 a expliqué se sentir \citeparticipant{impuissante} lorsqu'elle ne peut pas obtenir ce qu'elle veut dans un éditeur WYSIWYG.
À l'exception de P7, aucun de ceux qui connaissaient le mode \emph{rich text} d'Overleaf ne l'utilisent.
Plusieurs participants lui reprochent de manquer de fonctionnalités en cachant le code, impliquant ainsi \citeparticipant[P8]{plus de manipulation, plus de mouvement sur l'écran} afin de régulièrement passer d'un mode à l'autre.
P7 a d'ailleurs reconnu qu'elle l'avait uniquement utilisé pour prendre des notes pendant un cours d'histoire : \citeparticipant[P7]{il fallait juste taper ; j'avais pas de maths à faire ; j'avais pas de photo à rajouter}.

\paragraph{Le document doit pouvoir être programmé}
Plusieurs participants parmi les plus experts ont expliqué qu'ils aimaient pouvoir créer des commandes personnalisées pour générer ou réutiliser du contenu (\eg{} développer des abréviations, réutiliser un dessin paramétré).
Au fil des années, P11 s'est par exemple constitué un long préambule contenant plusieurs dizaines de commandes personnalisées pour générer des expressions mathématiques (telles que les noms de théorèmes et des constructions courantes), qu'il continue d'utiliser.
P10 explique avoir choisi \LaTeX{} pour créer une archive unique de centaines d'examens de philosophie de lycée, car il souhaitait pouvoir générer plusieurs index (par auteur, par thème, par type de question philosophique, etc) qui seraient automatiquement mis à jour s'il ajoutait de nouveaux examens par la suite.
Selon lui, un tel niveau d'automation est impossible à atteindre avec un système de préparation de document de type WYSIWYG : \citeparticipant[P10]{sous LibreOffice, on ne peut avoir un index alphabétique que pour une seule catégorie}.
La nature textuelle de \LaTeX{} permet également de générer des documents de toute pièce facilement.
P5 a ainsi écrit un programme Python pour générer un code \LaTeX{} représentant plusieurs centaines de badges similaires pour un concours de Go : \citeparticipant[P5]{Si vous me demandez de vous faire deux cent badges en Word, vous vous démer\textasteriskcentered\textasteriskcentered\textasteriskcentered{}~; je le fais pas}.

\paragraph{Les commentaires ont de multiples utilités}
Seule une fraction des participants a déclaré utiliser des commentaires dans le code ; mais ceux qui s'en servent expliquent que leur utilité est multiple.
Les raisons de s'en servir incluent
(1) se rappeler de l'utilité d'un paquet ou d'une commande ;
(2) discuter avec d'autres co-auteurs ;
(3) conserver des morceaux de code inutilisés qui pourraient servir plus tard ;
(4) commenter le code afin de trouver la source d'une erreur ;
(5) réutiliser des extraits d'un ancien morceau de code pour en écrire une nouvelle version ;
ou encore (6) planifier le contenu à écrire dans chaque partie du document.

\paragraph{L'accès au code facilite la réutilisation}
P5 a expliqué que le fait d'avoir accès au code permettait de copier-coller immédiatement des solutions trouvées sur internet, sans avoir à reproduire les étapes énumérées dans un tutoriel destiné aux éditeurs WYSIWYG.
Cet accès lui permet également de réutiliser le code \LaTeX{} de ses articles scientifiques afin de créer des présentations pour des colloques.



\subsubsection{T2 --- Apprendre \LaTeX{} est compliqué et chronophage}

\paragraph{\LaTeX{} est difficile à apprendre}
La plupart des participants ont expliqué qu'ils attendaient généralement de rencontrer des problèmes pour se renseigner sur \LaTeX{} et ses fonctionnalités.
Plusieurs ont souligné que cela arrivait souvent dans des moments de hâte, peu avant un rendu --- ne leur laissant donc pas la possibilité de prendre le temps de comprendre le code qu'ils devaient utiliser (après l'avoir copié-collé) : \citeparticipant[P2]{Pour l'instant je suis plus dans l'optique que ça doit marcher}.
Le nom des commandes, l'ordre de leurs paramètres, et les valeurs autorisées pour ces derniers semblent souvent être oubliés ou ignorés par les participants --- y compris lorsqu'ils ont déjà l'habitude de s'en servir : \citeparticipant[P3]{Je saurais qu'il faut une minipage, mais je serais quand même obligé d'aller chercher quelque part comment l'écrire}.

\paragraph{Ne pas comprendre \LaTeX{} est coûteux}
Certains participants ont reconnu que leur incompréhension de \LaTeX{} était parfois coûteuse : \citeparticipant[P7]{[Personnaliser] la biblio, c'est le truc qui était le plus long, parce que j'ai pas trouvé de solution toute faite quoi. Il a fallu que j'invente un petit peu \elips{}}.
P2 et P4 ont exprimé des regrets de n'avoir jamais eu de cours sur \LaTeX{} avant de devoir l'utiliser : \citeparticipant[P2]{J'aurais voulu prendre une journée [pour apprendre \LaTeX{}] ; il y a des formations qui présentent les bases}.
P6 a au contraire expliqué qu'il ne voudrait pas faire un tel effort : \citeparticipant[P6]{Tu trouves un document qui marche, tu copies-colle, et tu changes par itération \elips{} je vais pas apprendre la structure du code.}

\paragraph{Les recherches portent sur des problèmes spécifiques}
Les participants ont indiqué qu'ils cherchaient souvent des solutions à des problèmes spécifiques : produire un symbole particulier, comprendre une erreur, trouver la documentation d'un paquet, etc.
Lorsqu'il a eu besoin de poursuivre la numérotation d'une liste, P3 a par exemple rapidement trouvé la solution sur internet : \citeparticipant[P3]{j'ai trouvé que c'était après le \texttt{\symbol{92}begin\string{enumerate\string}} qu'il fallait mettre le start, égal, ce qu'on veut}.
Néanmoins, trouver une réponse ne semble pas toujours être évident pour tous les participants.
Afin de trouver une solution pour insérer du grec ancien dans son document \LaTeX{}, P8 a par exemple expliqué avoir \citeparticipant[P8]{plutôt cherché sur des trucs de linguistique et d'historien} --- où elle n'a pas pu trouver de réponse.



\subsubsection{T3 --- Les éditeurs textuels sont inadaptés au contenu structuré}

\paragraph{Les éditeurs de code ignorent les structures}
De nombreux participants se sont plaints de la difficulté de décrire des éléments structurés tels que des tableaux, des sous-figures, ou même des formules chimiques.
La création de tableaux a souvent été décrite comme \citeparticipant[P1]{vachement chi\textasteriskcentered\textasteriskcentered\textasteriskcentered{}}, et la difficulté de leur édition leur a été reprochée : \citeparticipant[P3]{ça m'est arrivé de devoir déplacer des colonnes \elips{} ça se fait, mais c'est juste que ça prend du temps}.
Selon P7, à cause de la syntaxe des tableaux, \citeparticipant{oublier une colonne c'est un enfer} car \citeparticipant[P7]{il faut que je rajoute des trucs et qu'à chaque ligne je compte, je me mette au bon endroit}.
Ces observations s'appliquent également à d'autre types de contenu structuré, en particulier lorsque les participants ne sont pas familiers avec un paquet offrant une solution clé en main : \citeparticipant[P4]{L'autre truc qui est vraiment difficile aussi, c'est de mettre les images côte à côte}.

\paragraph{Le recours aux outils externes}
En raison des difficultés à créer et modifier du contenu structuré depuis leurs éditeurs \LaTeX{}, plusieurs participants se tournent vers des outils externes.
P2 et P4 ont par exemple toutes deux commencé par utiliser des sous-figures constituées d'images séparées dans \LaTeX{} avant de finalement les fusionner en utilisant Inkscape ou Adobe Illustrator car \citeparticipant[P2]{comme ça je maîtrisais la mise en page}.
Afin de créer des tableaux à partir de classeurs Excel, P7 les a exportés au format CSV afin de générer le code \LaTeX{} correspondant en utilisant le convertisseur en ligne \url{tablesgenerator.com}.
Elle a cependant expliqué avoir dû répéter le processus à chaque fois qu'elle souhaitait modifier les tableaux, dont le code généré était trop difficile à interpréter sous forme de texte brut.
D'autres participants ont par ailleurs indiqué qu'ils utilisaient ce même site web pour créer des tableaux \LaTeX{} à partir de zéro.
P7 s'est également plainte de ne pas pouvoir exporter les structures créées dans des logiciels spécialisés tels que Microsoft Excel (pour les tableaux) ou ChemDraw (pour les molécules) sous forme de code \LaTeX{} --- soulignant là encore à quel point les éditeurs de code sont inadaptés pour décrire et manipuler certaines structures :
\citeparticipant{un truc qui serait utile c'est de dessiner ma molécule de A à Z et que le code s'affiche à côté quoi}, car \citeparticipant[P7]{le truc le plus difficile que j'ai jamais utilisé sous \LaTeX{} c'est le package pour faire les molécules chimiques}.



\subsubsection{T4 --- Les abstractions sont difficiles à visualiser et à formaliser}

\paragraph{Les dimensions}
De nombreuses commandes \LaTeX{} requièrent de spécifier des dimensions, comme par exemple la taille d'une image ou les marges de certains éléments.
Or, selon plusieurs participants, les dimensions de \LaTeX{} souffrent de deux problèmes :
il est difficile de quantifier une longueur que l'on imagine, et il est difficile d'imaginer une longueur quantifiée.
\citeparticipant[P8]{Je ne sais pas forcément quelle taille précise je veux que l'image ait, mais [je sais] que je veux qu'elle ait cette taille-là dans ma tête}.
Comme de nombreux autres participants, P1 s'est aussi plainte de devoir essayer plusieurs dimensions jusqu'à trouver celle qui convient --- une stratégie parfois coûteuse : \citeparticipant[P1]{quand tu veux que [tes figures] soient un peu plus grosses, un peu plus petites, tu compiles quinze mille fois et c'est chi\textasteriskcentered\textasteriskcentered\textasteriskcentered{}~\elips{} surtout quand tu as beaucoup de figures, et que du coup il met beaucoup de temps à compiler}.
La plupart des participants semblent donc favorables à une forme d'interaction plus directe avec les dimensions.
Afin d'outrepasser ces difficultés, certains participants ont fait le choix de détourner des commandes avec lesquelles ils se sentent plus familiers.
P3 a ainsi inséré du texte de couleur blanche dans son document afin de sauter \citeparticipant{artificiellement} plusieurs lignes successives : \citeparticipant[P3]{je pense qu'il y avait une façon plus simple de le faire, mais comme j'étais assez pressé c'est comme ça}.

\paragraph{Les positions}
Au contraire des dimensions, \LaTeX{} se charge traditionnellement de calculer les positions des éléments à la place de l'utilisateur.
Cependant, selon certains participants, ce manque de contrôle n'est pas toujours souhaitable.
P1 a par exemple expliqué que sa façon de positionner les figures n'était \citeparticipant[P1]{pas très rationnelle}, et P2 s'est plainte de la difficulté de placer une image suffisamment près d'un paragraphe y faisant référence lorsque \LaTeX{} en décidait autrement.
Ces difficultés sont probablement renforcées par le fait que plusieurs participants ne sont pas conscients du sens des paramètres de positionnement qu'ils utilisent (\eg{} avec l'environnement \texttt{figure}) --- souvent copiés avec le reste d'un morceau de code.
Par conséquent, P4 --- qui a pourtant lu \citeparticipant{comment le compilateur faisait pour placer les images} afin de gagner en contrôle sur leur positionnement --- a ainsi admis avoir dû revoir ses attentes concernant les positions des figures à la baisse.
Ces difficultés concernent également les positions plus locales.
Bien qu'il se sente à l'aise avec les commandes de dessin du paquet TikZ, P5 a expliqué qu'il aimerait bien pouvoir déplacer certains éléments de ses dessins au lieu \citeparticipant{d'essayer de deviner} les bonnes coordonnées ou de faire \citeparticipant{vaguement de la trigo} pour les calculer.



\subsubsection{T5 --- La dualité entre le code et le PDF est coûteuse}

\paragraph{Identifier le code d'un élément du PDF est difficile}
Certains participants se sont plaints de la difficulté de faire le lien entre un élément du PDF et le code l'ayant généré.
P5 et P9 ont tous deux du mal à (1) localiser le code d'une formule mathématique affichée dans le PDF et (2) trouver le symbole qu'ils veulent corriger à l'intérieur du code de celle-ci.
Afin de résoudre le premier problème, ils recherchent souvent un extrait du texte voisin de la formule dans leurs éditeurs de code, bien que P5 ait admis que cette technique échoue régulièrement.
Cette approche n'est pas sans lien avec l'absence de support de SyncTeX dans l'éditeur \LaTeX{} que P5 utilise.
Il a d'ailleurs expliqué qu'il aimerait en changer pour cette raison --- soulignant l'importance de ce genre d'outil.
En revanche, ni P5 ni P9 n'ont trouvé de solution au second problème : pour eux, modifier une formule mathématique implique systématiquement \citeparticipant[P9]{de relire le code \LaTeX{} sans relire le PDF}.

\paragraph{Compiler est un processus lent}
De nombreux participants se sont plaints du temps requis pour compiler leurs documents \LaTeX{}.
Essayer plusieurs alternatives (\eg{} différentes tailles pour une image) est ainsi coûteux, et les erreurs font perdre du temps : \citeparticipant[P5]{si on se trompe ça nous coûte quand même une minute}.
Certains participants ont élaboré des stratégies en conséquence.
Afin de compiler moins souvent, P1 a expliqué qu'elle distinguait rédaction et mise en forme : \citeparticipant[P1]{quand je suis dans une phase de rédaction \elips{} je tape juste et j'écris \elips{} mais quand je fais la forme j'ouvre toujours [l'éditeur de code et le PDF] à côté, pour compiler régulièrement}.
P5 a également donné l'exemple de la mise en cache d'images crées avec TikZ : \citeparticipant[P5]{tant qu'il n'y a pas de modification faite sur l'image, [l'outil] réinjecte l'image à la place du code TikZ}.

\subsection{Recommandations pour le design}
Cette analyse thématique révèle la variété des problèmes auxquels les utilisateurs de \LaTeX{} font face et suggère plusieurs opportunités pour améliorer le design des éditeurs de documents \LaTeX{}. Nous résumons celles-ci dans les quatre recommandations suivantes.

\paragraph{Conserver l'accès au code}
Nous considérons qu'aucune interface graphique ne peut complètement dissimuler le code d'un document \LaTeX{} sans brider ses utilisateurs (\eg{} en ne supportant pas certaines fonctionnalités). 
Nous recommandons donc de permettre aux utilisateurs de \LaTeX{} de lire et de modifier librement et facilement le code de leurs documents.
De plus, certains utilisateurs veulent pouvoir réutiliser du code facilement ou automatiser sa génération.

\paragraph{Cibler les problèmes spécifiques et courants}
Même si la majorité des participants n'était pas fermée à l'idée d'apprendre à utiliser \LaTeX{}, aucun d'entre eux ne peut ou ne veut y consacrer beaucoup de temps.
Nous recommandons donc de ne pas restreindre les éditeurs \LaTeX{} à des fonctionnalités génériques plutôt destinées aux experts (\eg{} l'auto-complétion de commandes, qu'il faut donc connaître), mais également de prendre en compte les problèmes et les besoins plus spécifiques mais courants (\eg{} tableaux, sous-figures, références) et d'aider les utilisateurs moins expérimentés et intermittents à trouver ou concevoir des solutions à leurs problèmes.

\paragraph{Donner forme aux abstractions invisibles}
Étant donné la nature entièrement textuelle de \LaTeX{}, nous recommandons d'aider les utilisateurs à (1) visualiser les structures et les abstractions codées sous forme de texte lorsqu'elles s'y prêtent et (2) manipuler ces dernières à l'aide d'interactions adaptées tout en maintenant la synchronisation du texte source.
Ces visualisations doivent viser à réduire le nombre de cycles de compilation en facilitant la constitution de représentations mentales du document.

\paragraph{Renforcer les liens entre code source et document généré}
Face à la dualité entre code et rendu d'un même document, nous recommandons d'améliorer la visibilité et la spécificité des liens qui unissent ces deux substrats (\eg{} faciliter l'identification et la modification de la commande responsable d'un symbole mathématique particulier dans le PDF).

Les représentations intermédiaires interactives nous semblent pouvoir répondre à certains des problèmes soulevés par cette analyse thématique en respectant les recommandations que nous proposons ci-dessus.
Nous avons ainsi développé \iLaTeX, un prototype d'éditeur \LaTeX{} doté de RII, que nous présentons dans la section suivante.