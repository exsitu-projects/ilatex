% !TEX root = main.tex

\section{Experimental Details: Sequence Modeling}

\begin{table}[!h]
	\centering
	% \begin{threeparttable}
		\begin{itabular}{lccc}
			\hline
			task & optimizer & initial step-size & warmup epochs \\
			\hline
			Permuted MNIST & Adam & 2E-4 & 0 \\
			JSB Chorales & Adam & 2E-4 & 25 \\
			Nottingham & Adam & 2E-3 & 0 \\
			Penn Treebank & Adam & 2E-6 & 0 \\
			\hline
		\end{itabular}
		%		\begin{tablenotes}
		%			\item[$\ast$] table note 1
		%			\item[$\dagger$] table note 2
		%		\end{tablenotes}
		\caption{\label{app:tab:seq}
			Architecture optimizer settings on sequence modeling tasks.
			Note that the step-size is updated using the same schedule as the backbone.
		}
	% \end{threeparttable}
\end{table}

For our sequence modeling experiments we use the code of \citet{bai2018tcn} provided here: \url{https://github.com/locuslab/TCN}.
As before we use the same settings and training routine as the backbone for all tasks, tuning only the architecture optimizer.
The specific settings are provided in Table~\ref{app:tab:seq}.