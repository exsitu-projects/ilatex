% Task 1

\renewcommand{\taskone}{
    \newtask
    
    \begin{instructions}
        Replace $\varphi_j$ by $\varphi_k$ in the bottom-left value of the matrix (between the square brackets).
    \end{instructions}
}




% Task 2

\renewcommand{\tasktwo}{
    \newtask
    
    \begin{instructions}
        Replace ``$n-1$'' by ``$n-2$'' in the product (~$\prod$~) in the denominator of the fraction on the 5th line.
    \end{instructions}
}




% Task 3

\renewcommand{\taskthree}{
    \newtask
    
    \begin{instructions}
        Replace the last ``$f_{0}$'' factor by ``$f_{-1}$'' in the numerator of the fraction in equation (2).
    \end{instructions}
}




% Task 4

\renewcommand{\taskfour}{
    \newtask
    
    \begin{instructions}
        Sort the rows of the table by the values in the ``ArIc(in \%)'' column in \emph{descending} order (maximum in the first row, minimum in the last row).
    \end{instructions}
}




% Task 5

\renewcommand{\taskfive}{
    \newtask
    
    \begin{instructions}
    Replace the values in the following cells (keep the same uncertainty):
    
    \begin{center}
        \begin{tabular}{llr}
            \textbf{Row} & \textbf{Column} & \textbf{New value} \\
            \midrule
            DGK & REDDIT-12K & $34.08$ \\
            GIN-GFL & REDDIT-B & $91.44$ \\
            P-Hybrid & IMDB-B & $73.48$ \\
        \end{tabular}  
    \end{center}
    \end{instructions}
}




% Task 6

\renewcommand{\tasksix}{
    \newtask
    
    \begin{instructions}
        Remove the ``$\mu_i$'' column.
    \end{instructions}
}




% Task 7

\renewcommand{\taskseven}{
    \newtask
    
    \begin{instructions}
        % Resize both images to make their widths match the width of the coloured box they correspond to.
        Resize the image so that it has the same width than the coloured box displayed below.
    \end{instructions}
}




% Task 8

\renewcommand{\taskheight}{
    \newtask
    
    \begin{instructions}
        The image used in the \texttt{wrapfigure} environment has a lot of whitespace around its content.
        Make the necessary changes so that the black frame within the image
        \begin{itemize}
            \item has the same width than the black line below;
            \item is aligned with the the black line below.
        \end{itemize}
    \end{instructions}
}




% Task 9

\renewcommand{\tasknine}{
    \newtask
    
    \begin{instructions}
        Hide the title of the graph (``Scores by group and gender'').
    \end{instructions}
}




% Task 10

\renewcommand{\taskten}{
    \newtask
    
    \begin{instructions}
        Resize the cells of the layout so that
        \begin{itemize}
            \item both image have the same height (the horizontal black lines must be aligned);
            \item the two cells span over all the horizontal space of the row.
        \end{itemize}
    \end{instructions}
}





% Task 11

\renewcommand{\taskeleven}{
    \newtask
    
    \begin{instructions}
        Reorganise the cells to make the images form the following pattern (on two rows instead of three): \\[1ex]
        
        \centering
        \begin{tabular}{rrcrrcrr}
            1 & 2 & ~ & 5 & 6 & ~ & 9 & 10 \\
            3 & 4 & ~ & 7 & 8 & ~ & 11 & 12
        \end{tabular} \\[1em]
        
        All the images must have the same size and there must be no overlapping.
    \end{instructions}
}





% Task 12

\renewcommand{\tasktwelve}{
    \newtask
    
    \begin{instructions}
        Make the necessary changes so that each logo of this fake front cover
        \begin{itemize}
            \item has the same width than the colored box it corresponds to;
            \item is aligned with the colored box it corresponds to.
        \end{itemize}
    \end{instructions}
}

