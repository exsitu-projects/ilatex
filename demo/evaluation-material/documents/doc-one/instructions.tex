% Task 1

\renewcommand{\taskone}{
    \newtask{1}
    
    \begin{instructions}
        Insert $\displaystyle\frac{x-1}{x(x+1)},$ between  $\displaystyle\frac{x(x+1)}{x-1},$ and  $\displaystyle\frac{x(x+1)}{x^2+1}$ on the 2nd line.
    \end{instructions}
}




% Task 2

\renewcommand{\tasktwo}{
    \newtask{2}
    
    \begin{instructions}
        Remove the parentheses around the last term of the product on the the 5th and 6th lines.
    \end{instructions}
}




% Task 3

\renewcommand{\taskthree}{
    \newtask{3}
    
    \begin{instructions}
        Replace the \textbf{second} $f_{0}$ factor by $f_{-1}$ in the numerator of the fraction in equation (2).
    \end{instructions}
}




% Task 4

\renewcommand{\taskfour}{
    \newtask{4}
    
    \begin{instructions}
        Sort the rows of the table by the values in the ``ArIc(in \%)'' column in \emph{descending} order (maximum in the first row, minimum in the last row).
    \end{instructions}
}




% Task 5

\renewcommand{\taskfive}{
    \newtask{5}
    
    \begin{instructions}
    Replace the values in the following cells (keep the same uncertainty):
    
    \begin{center}
        \begin{tabular}{llr}
            \textbf{Row} & \textbf{Column} & \textbf{New value} \\
            \midrule
            DGK & REDDIT-12K & $34.08$ \\
            GIN-GFL & REDDIT-B & $91.44$ \\
            P-Hybrid & IMDB-B & $73.48$ \\
        \end{tabular}  
    \end{center}
    \end{instructions}
}




% Task 6

\renewcommand{\tasksix}{
    \newtask{6}
    
    \begin{instructions}
        Remove the ``$\mu_i$'' column.
    \end{instructions}
}




% Task 7

\renewcommand{\taskseven}{
    \newtask{7}
    
    \begin{instructions}
        Resize the image so that it has the same width as the coloured box displayed below.
    \end{instructions}
}




% Task 8

\renewcommand{\taskheight}{
    \newtask{8}
    
    \begin{instructions}
        The image used in the \texttt{wrapfigure} environment has a lot of whitespace around its content.
        Make the necessary changes so that the black frame within the image
        \begin{itemize}
            \item has the same width as the black line below;
            \item is aligned with the the black line below.
        \end{itemize}
    \end{instructions}
}




% Task 9

\renewcommand{\tasknine}{
    \newtask{9}
    
    \begin{instructions}
        Hide the title of the graph (``Scores by group and gender'').
    \end{instructions}
}




% Task 10

\renewcommand{\taskten}{
    \newtask{10}
    
    \begin{instructions}
        Resize the cells of the layout so that
        \begin{itemize}
            \item both image have the same height (the horizontal black lines must be aligned);
            \item the two cells span over all the horizontal space of the row.
        \end{itemize}
    \end{instructions}
}





% Task 11

\renewcommand{\taskeleven}{
    \newtask{11}
    
    \begin{instructions}
        Reorganise the cells to make the images form the following pattern (on two rows instead of three): \\[1ex]
        
        \centering
        \begin{tabular}{cccccc}
            1 & 3 & 5 & 7 & 9 & 11 \\
            2 & 4 & 6 & 8 & 10 & 12
        \end{tabular} \\[1em]
        
        All the images must have the same size and there must be no overlapping.
    \end{instructions}
}





% Task 12

\renewcommand{\tasktwelve}{
    \newtask{12}
    
    \begin{instructions}
        Make the necessary changes so that each logo of this fake front cover
        \begin{itemize}
            \item has the same width as the colored box below it;
            \item is aligned with the colored box below it.
        \end{itemize}
    \end{instructions}
}

