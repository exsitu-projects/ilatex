\usepackage{inconsolata} % Monospace font
\usepackage{helvet} % Sans-serif font
\usepackage[
    libertine, % do not override the sans-serif-font
    tt = false % do not override the monospace font
]{libertine} % Main font (serif)

\usepackage[
    top = 3cm,
    bottom = 3cm,
    left = 2.5cm,
    right = 2.5cm
]{geometry}

\usepackage{xcolor}
\usepackage{graphicx}
\definecolor{DocPurple}{HTML}{BC43F0}

\usepackage[english]{babel}
\usepackage{xspace}
\usepackage{nopageno}
\usepackage{booktabs}
\usepackage{enumitem}
\usepackage[xcolor]{mdframed}
\usepackage{amsmath}
\usepackage{amssymb}
\usepackage{amsthm}
\usepackage{mathrsfs}
\usepackage{stmaryrd}
\usepackage{float}
\usepackage{ragged2e}
\usepackage{listings}
\usepackage{parskip}
\usepackage[
    labelfont = {small, bf},
    textfont = {small}
]{caption}
\usepackage{circledsteps}
\usepackage[hidelinks]{hyperref}

\usepackage{ilatex}

% ====================================================================

\newcommand{\eg}{e.g.,\xspace}
\newcommand{\ie}{i.e.\xspace}
\newcommand{\etal}{\textit{et~al.}\xspace}


\DeclareRobustCommand{\iLaTeX}{\mbox{{{\itshape i}-\hspace{-0.25mm}}\LaTeX{}}}

% Document titles
\newcommand{\makedoctitle}[1]{%
\begin{center}
    \color{black!75}
    {\Large{Evaluation of \iLaTeX}} \\
    \noindent\rule{8cm}{0.4pt} \\[1em]
    \color{black}
    {\Huge{#1}}
\end{center}
}

% Task titles
\newcommand{\largetitle}[1]{
    \begin{center}
        \Huge
        \textbf{#1}
    \end{center}
    \vspace{2em}
}

% Various "blocks" defined with mdframed
\newmdenv[
    linecolor = black!50,
    backgroundcolor = white,
    skipabove = 1em,
    skipbelow = 1em,
    frametitle = {Instructions},
    frametitlebackgroundcolor = black!5
]{instructions}

\newmdenv[
    linecolor = blue!50!black,
    backgroundcolor = white,
    skipabove = 1em,
    skipbelow = 1em,
    innertopmargin = 1em,
    innerbottommargin = 1em,
    frametitle = {What you must edit},
    frametitlebackgroundcolor = blue!5,
    startinnercode = \centering\bgroup,
    endinnercode = \egroup
]{task}

\newmdenv[
    linecolor = red!50!black,
    backgroundcolor = red!2,
    skipabove = 1em,
    skipbelow = 1em,
    innertopmargin = 1em,
    innerbottommargin = 1em,
    frametitle = {Warning},
    frametitlebackgroundcolor = red!5,
    % startinnercode = \centering\bgroup,
    % endinnercode = \egroup
]{warning}

\newmdenv[
    linecolor = blue!50!black,
    backgroundcolor = blue!2,
    skipabove = 1em,
    skipbelow = 1em,
    innertopmargin = 1em,
    innerbottommargin = 1em,
    frametitle = {Good to know},
    frametitlebackgroundcolor = blue!5,
    % startinnercode = \centering\bgroup,
    % endinnercode = \egroup
]{info}

\newmdenv[
    linecolor = green!50!black,
    backgroundcolor = green!2,
    skipabove = 1em,
    skipbelow = 1em,
    innertopmargin = 1em,
    innerbottommargin = 1em,
    frametitle = {Example},
    frametitlebackgroundcolor = green!5,
    % startinnercode = \centering\bgroup,
    % endinnercode = \egroup
    nobreak
]{example}

% Style of code blocks
\lstdefinestyle{custom-latex}{
    language={[LaTeX]TeX},
    backgroundcolor=\color{white},
    commentstyle=\color{black!50},
    keywordstyle=\color{green!50!black},
    numberstyle=\tiny\color{red!50!black},
    stringstyle=\color{blue!50!black},
    basicstyle=\ttfamily\small,
    breakatwhitespace=false,         
    breaklines=true,
    keepspaces=true,
    numbers=none,
    tabsize=4,
    aboveskip={1em},
    belowskip={0.7em}
}

% Global list configuration
\setlist{noitemsep}

% "Tasks"
\newcommand{\taskone}{
    \begin{instructions}
        Reorganise the cells to make the images form the following pattern (on two rows instead of three): \\[1ex]
        
        \centering
        \begin{tabular}{cccccc}
            1 & 3 & 5 & 7 & 9 & 11 \\
            2 & 4 & 6 & 8 & 10 & 12
        \end{tabular} \\[1em]
        
        All the images must have the same size and there must be no overlapping.
    \end{instructions}
}

\newcommand{\tasktwo}{
    \begin{instructions}
        Make the necessary changes so that each logo of this fake front cover
        \begin{itemize}
            \item has the same width as the colored box below it;
            \item is aligned with the colored box below it.
        \end{itemize}
    \end{instructions}
}

% Command to produce a circled number to reference a step represented in a figure
\DeclareRobustCommand{\figstep}[1]{%
    \Circled[%
        inner color=white,%
        outer color=white,%
        fill color=DocPurple%
    ]{\sffamily\textbf{#1}}%
}