\section{Conclusion}
\label{sec:conclusion}

Nous avons présenté le concept de représentations intermédiaires interactives, un nouveau genre d'interface utilisateur pour les éditeurs de langages de description de document.
Nous avons motivé son utilité en réalisant une analyse thématique des difficultés rencontrées par 11 utilisateurs de \LaTeX{} que nous avons interviewés, et nous l'avons également mis en œuvre à travers \iLaTeX{}, un éditeur \LaTeX{} expérimental doté de RII.
\iLaTeX{} a reçu des critiques majoritairement positives lors d'une évaluation préliminaire auprès des participants que nous avions interviewés, qui nous encouragent donc à poursuivre ce travail.

Parmi les directions possibles, quatre d'entre elles nous semblent particulièrement intéressantes à explorer.
Premièrement, une évaluation plus complète de \iLaTeX{} nous paraît primordiale.
Nous pensons qu'une étude de terrain auprès d'utilisateurs de \iLaTeX{} serait la plus adaptée pour évaluer le prototype de manière plus approfondie.
Deuxièmement, une analyse des besoins spécifiques de certains utilisateurs (\eg{} formules mathématiques, molécules) permettrait de doter les visualisations existantes de fonctionnalités supplémentaires adaptées et de développer de nouvelles visualisations répondant à d'autres besoins.
Troisièmement, développer des métriques spécialisées permettrait de mieux estimer la robustesse de \iLaTeX{} (\eg{} le taux de succès de l'analyse syntaxique sur un corpus de fichiers \LaTeX{}\footnote{L'évaluation pourrait par exemple porter sur les archives des sources \LaTeX{} des articles d'un laboratoire, ou encore sur celles publiées sur une plate-forme en ligne comme arXiv (\url{https://arxiv.org/}).}) et l'utilité potentielle de ses visualisations (\eg{} la proportion de documents \LaTeX{} contenant du code susceptible d'être visualisé).
Quatrièmement, nous pensons qu'il est important d'appliquer le concept de RII à d'autres langages de description de documents que \LaTeX{} (\eg{} adapter la visualisation de tableau afin de l'utiliser avec des tableaux HTML ou Markdown).
Cela permettrait non seulement de mieux comprendre le degré de généricité de ce type d'interface utilisateur, mais également le potentiel de réutilisation de ses visualisations.